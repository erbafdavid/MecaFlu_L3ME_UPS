% !TEX root = TD_fluides_part2_2018.tex



\setcounter{section}{11}

\section{Ondes de choc}


%--------------------------------------------------------------------------------------------------
\subsection{Tube de Pitot en \'ecoulement supersonique}
%--------------------------------------------------------------------------------------------------

On considère un \'ecoulement supersonique uniforme dont on cherche à mesurer les caract\'eristiques
(vitesse $u_1$, nombre de Mach $M_1$, pression $p_1$, pression g\'en\'eratrice $p_{i1}$) 
à l'aide d'un tube de Pitot double.
\begin{enumerate}
\item
D\'ecrire qualitativement l'\'ecoulement dans le voisinage du tube de Pitot.
\item
D\'eterminer l'expression du nombre de Mach $M_2$ au niveau du tube de Pitot en fonction
des pressions $p_2$ et $p_{i2}$ mesur\'ees par le tube de Pitot.
\item
En d\'eduire $M_1$.
\item
D\'eterminer $p_1$ et $p_{i1}$.
\item
Expliquer comment il serait possible de pr\'edire la vitesse $u_1$ de l'\'ecoulement incident. 
\end{enumerate}
 
%--------------------------------------------------------------------------------------------------
\subsection{R\'egime d'\'ecoulement discontinu dans une tuyère}
%--------------------------------------------------------------------------------------------------

On considère un \'ecoulement dans une tuyère caract\'eris\'e par la pr\'esence d'une onde de choc.
On souhaite d\'ecrire ce r\'egime particulier de fonctionnement de la tuyère.   
\begin{enumerate}
\item
Pr\'eciser dans quelle partie de la tuyère se trouve cette onde de choc.
Peut-il exister une autre onde de choc dans la tuyère ?
\item
Donner le nombre de Mach au col $M_c$.
\item
Tracer pour un tel r\'egime l'allure de la vitesse, de la pression et du nombre de Mach
le long de la tuyère.
\item
Que devient l'onde de choc lorsque la pression en sortie $p_s$ diminue ?
\item 
D\'eterminer en fonction des conditions g\'en\'eratrices en amont la plage
de pression $p_s$ correspondant à ce r\'egime d'\'ecoulement discontinu dans la tuyère.
\end{enumerate}
 
%--------------------------------------------------------------------------------------------------
\subsection{Onde de choc dans une tuyère}
%--------------------------------------------------------------------------------------------------

Une tuyère convergente-divergente est aliment\'ee avec de l'air à pression
g\'en\'eratrice $p_i$ de 3 bars et une temp\'erature $T_i$ de 600 K.
Cette tuyère a une section d'entr\'ee $A_e$ de 5 cm$^2$ dans laquelle la vitesse
$u_e$ a pour valeur 146 m.s$^{-1}$.
On constate lors du fonctionnement la pr\'esence d'une onde de choc dans
la section $A\indice{choc}$ d'aire 2.53 cm$^2$.
L'objectif de cet exercice est d'utiliser les relations isentropiques et de saut afin
de d\'eterminer certaines caract\'eristiques de l'\'ecoulement \'etudi\'e.
\begin{enumerate}
\item
  Calculer la pression $p_c$, la temp\'erature $T_c$ et la masse volumique $\rho_c$ au col ainsi
  que le d\'ebit massique $\dot{m}$ de la tuyère.
\item
  Calculer les pressions $p_1$ et $p_2$ et les temp\'eratures $T_1$ et $T_2$
  imm\'ediatement en amont et en aval de l'onde de choc.
\item
  Calculer la pression $p_s$ et la temp\'erature $T_s$ dans la section de sortie $A_s$ d'aire
  2.70 cm$^2$.
\end{enumerate}

 
 
 
 