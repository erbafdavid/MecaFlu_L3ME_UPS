\documentclass[12pt,a4]{article}
\usepackage[french]{babel}
\usepackage[T1]{fontenc} 

\usepackage{floatflt,fancyheadings,amssymb,color}
\usepackage{psfig,epsfig}
\usepackage{pictex}
\usepackage{charter}
\hoffset=-4cm 
\voffset=-4cm
\topmargin=15mm
\headheight=5mm
\headsep=15mm
\textheight=260mm
\oddsidemargin=30mm
\evensidemargin=30mm
\textwidth=170mm
\pagestyle{plain}
\renewcommand{\textfraction}{.10}
\newcommand{\derp}[2]{ \frac{ \partial #1 }{ \partial #2}}
\newcommand{\dsp} {\displaystyle}
\newcommand{\torseur}[4]{
\Biggl\{ {\cal #1} (#2)\Biggr \} = 
\left \{ \begin{array}{c}
#3  \\
#4 
\end{array} \right\} 
}
    
    

\begin{document}
\sf 


\begin{center}
  Licence II math�matiques \& M�canique, Universit\'e Paul Sabatier.
 \vspace{0.6cm}
 
Th�me 4 (suite)

 {\bf \Large Le diffuseur de parfum }


\documentclass[12pt,a4]{article}
\usepackage[french]{babel}
\usepackage[T1]{fontenc} 

\usepackage{floatflt,fancyheadings,amssymb,color}
\usepackage{psfig,epsfig}
\usepackage{pictex}
\usepackage{charter}
\hoffset=-4cm 
\voffset=-4cm
\topmargin=15mm
\headheight=5mm
\headsep=15mm
\textheight=260mm
\oddsidemargin=30mm
\evensidemargin=30mm
\textwidth=170mm
\pagestyle{plain}
\renewcommand{\textfraction}{.10}
\newcommand{\derp}[2]{ \frac{ \partial #1 }{ \partial #2}}
\newcommand{\dsp} {\displaystyle}
\newcommand{\torseur}[4]{
\Biggl\{ {\cal #1} (#2)\Biggr \} = 
\left \{ \begin{array}{c}
#3  \\
#4 
\end{array} \right\} 
}
    
    

\begin{document}
\sf 


\begin{center}
  Licence II math�matiques \& M�canique, Universit\'e Paul Sabatier.
 \vspace{0.6cm}
 
Th�me 4 (suite)

 {\bf \Large Le diffuseur de parfum }


\documentclass[12pt,a4]{article}
\usepackage[french]{babel}
\usepackage[T1]{fontenc} 

\usepackage{floatflt,fancyheadings,amssymb,color}
\usepackage{psfig,epsfig}
\usepackage{pictex}
\usepackage{charter}
\hoffset=-4cm 
\voffset=-4cm
\topmargin=15mm
\headheight=5mm
\headsep=15mm
\textheight=260mm
\oddsidemargin=30mm
\evensidemargin=30mm
\textwidth=170mm
\pagestyle{plain}
\renewcommand{\textfraction}{.10}
\newcommand{\derp}[2]{ \frac{ \partial #1 }{ \partial #2}}
\newcommand{\dsp} {\displaystyle}
\newcommand{\torseur}[4]{
\Biggl\{ {\cal #1} (#2)\Biggr \} = 
\left \{ \begin{array}{c}
#3  \\
#4 
\end{array} \right\} 
}
    
    

\begin{document}
\sf 


\begin{center}
  Licence II math�matiques \& M�canique, Universit\'e Paul Sabatier.
 \vspace{0.6cm}
 
Th�me 4 (suite)

 {\bf \Large Le diffuseur de parfum }


\documentclass[12pt,a4]{article}
\usepackage[french]{babel}
\usepackage[T1]{fontenc} 

\usepackage{floatflt,fancyheadings,amssymb,color}
\usepackage{psfig,epsfig}
\usepackage{pictex}
\usepackage{charter}
\hoffset=-4cm 
\voffset=-4cm
\topmargin=15mm
\headheight=5mm
\headsep=15mm
\textheight=260mm
\oddsidemargin=30mm
\evensidemargin=30mm
\textwidth=170mm
\pagestyle{plain}
\renewcommand{\textfraction}{.10}
\newcommand{\derp}[2]{ \frac{ \partial #1 }{ \partial #2}}
\newcommand{\dsp} {\displaystyle}
\newcommand{\torseur}[4]{
\Biggl\{ {\cal #1} (#2)\Biggr \} = 
\left \{ \begin{array}{c}
#3  \\
#4 
\end{array} \right\} 
}
    
    

\begin{document}
\sf 


\begin{center}
  Licence II math�matiques \& M�canique, Universit\'e Paul Sabatier.
 \vspace{0.6cm}
 
Th�me 4 (suite)

 {\bf \Large Le diffuseur de parfum }


\input{parfum.pstex_t}

\end{center}


Le dispositif repr�sent� sur la figure est utilis� pour diffuser du parfum.
Un courrant d'air, de d�bit volumique $q$ constant, circule dans un conduit
pr�sentant un r�tr�cissement. Un tuyau, raccord� au r�cipient contenant le 
parfum, est dispos� au col. Si le d�bit d'air est suffisant, le parfum
est aspir� dans le tuyau, et circule avec un d�bit volumique $q_l$.
On suppose le d�bit volumique du parfum faible devant celui de l'air,
de sorte que le d�bit volumique en sortie est $q+q_l \approx q$.

On note $\rho_a$ et $\rho_l$ les masses volumiques de l'air et du liquide,
$S$ et $S_c$ les sections en entr�e et au col, et $\Delta h$ la hauteur entre
la surface du liquide dans le r�cipient et la sortie du tuyau.

1/ Montrez qu'un d�bit d'air minimal est n�cessaire pour que le parfum
soit diffus�.
Lorsque le d�bit d'air est inf�rieur � cette valeur, donnez la hauteur atteinte
dans le tube par le parfum.

2/ Lorsque le d�bit d'air est sup�rieur � la valeur critique, calculez
le d�bit volumique de parfum diffus�.

3/ Repr�sentez la fraction volumique de parfum, $q_l/q$, en fonction de $q$.



Remarque : Un dispositif du m�me type est utilis� dans certains carburateurs
(mais ca sent moins bon).

\end{document}


\end{center}


Le dispositif repr�sent� sur la figure est utilis� pour diffuser du parfum.
Un courrant d'air, de d�bit volumique $q$ constant, circule dans un conduit
pr�sentant un r�tr�cissement. Un tuyau, raccord� au r�cipient contenant le 
parfum, est dispos� au col. Si le d�bit d'air est suffisant, le parfum
est aspir� dans le tuyau, et circule avec un d�bit volumique $q_l$.
On suppose le d�bit volumique du parfum faible devant celui de l'air,
de sorte que le d�bit volumique en sortie est $q+q_l \approx q$.

On note $\rho_a$ et $\rho_l$ les masses volumiques de l'air et du liquide,
$S$ et $S_c$ les sections en entr�e et au col, et $\Delta h$ la hauteur entre
la surface du liquide dans le r�cipient et la sortie du tuyau.

1/ Montrez qu'un d�bit d'air minimal est n�cessaire pour que le parfum
soit diffus�.
Lorsque le d�bit d'air est inf�rieur � cette valeur, donnez la hauteur atteinte
dans le tube par le parfum.

2/ Lorsque le d�bit d'air est sup�rieur � la valeur critique, calculez
le d�bit volumique de parfum diffus�.

3/ Repr�sentez la fraction volumique de parfum, $q_l/q$, en fonction de $q$.



Remarque : Un dispositif du m�me type est utilis� dans certains carburateurs
(mais ca sent moins bon).

\end{document}


\end{center}


Le dispositif repr�sent� sur la figure est utilis� pour diffuser du parfum.
Un courrant d'air, de d�bit volumique $q$ constant, circule dans un conduit
pr�sentant un r�tr�cissement. Un tuyau, raccord� au r�cipient contenant le 
parfum, est dispos� au col. Si le d�bit d'air est suffisant, le parfum
est aspir� dans le tuyau, et circule avec un d�bit volumique $q_l$.
On suppose le d�bit volumique du parfum faible devant celui de l'air,
de sorte que le d�bit volumique en sortie est $q+q_l \approx q$.

On note $\rho_a$ et $\rho_l$ les masses volumiques de l'air et du liquide,
$S$ et $S_c$ les sections en entr�e et au col, et $\Delta h$ la hauteur entre
la surface du liquide dans le r�cipient et la sortie du tuyau.

1/ Montrez qu'un d�bit d'air minimal est n�cessaire pour que le parfum
soit diffus�.
Lorsque le d�bit d'air est inf�rieur � cette valeur, donnez la hauteur atteinte
dans le tube par le parfum.

2/ Lorsque le d�bit d'air est sup�rieur � la valeur critique, calculez
le d�bit volumique de parfum diffus�.

3/ Repr�sentez la fraction volumique de parfum, $q_l/q$, en fonction de $q$.



Remarque : Un dispositif du m�me type est utilis� dans certains carburateurs
(mais ca sent moins bon).

\end{document}


\end{center}


Le dispositif repr�sent� sur la figure est utilis� pour diffuser du parfum.
Un courrant d'air, de d�bit volumique $q$ constant, circule dans un conduit
pr�sentant un r�tr�cissement. Un tuyau, raccord� au r�cipient contenant le 
parfum, est dispos� au col. Si le d�bit d'air est suffisant, le parfum
est aspir� dans le tuyau, et circule avec un d�bit volumique $q_l$.
On suppose le d�bit volumique du parfum faible devant celui de l'air,
de sorte que le d�bit volumique en sortie est $q+q_l \approx q$.

On note $\rho_a$ et $\rho_l$ les masses volumiques de l'air et du liquide,
$S$ et $S_c$ les sections en entr�e et au col, et $\Delta h$ la hauteur entre
la surface du liquide dans le r�cipient et la sortie du tuyau.

1/ Montrez qu'un d�bit d'air minimal est n�cessaire pour que le parfum
soit diffus�.
Lorsque le d�bit d'air est inf�rieur � cette valeur, donnez la hauteur atteinte
dans le tube par le parfum.

2/ Lorsque le d�bit d'air est sup�rieur � la valeur critique, calculez
le d�bit volumique de parfum diffus�.

3/ Repr�sentez la fraction volumique de parfum, $q_l/q$, en fonction de $q$.



Remarque : Un dispositif du m�me type est utilis� dans certains carburateurs
(mais ca sent moins bon).

\end{document}
