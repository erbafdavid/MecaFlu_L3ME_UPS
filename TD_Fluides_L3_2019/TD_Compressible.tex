% !TEX root = TD_fluides_part2_2018.tex


\setcounter{section}{10}


\section{Ecoulements compressibles}


%--------------------------------------------------------------------------------------------------
\subsection{Limite de l'hypoth\`ese incompressible}
%--------------------------------------------------------------------------------------------------

Un tube de Pitot double est plac\'e sur l'axe d'une canalisation de 10 cm
de diam\`etre contenant de l'air en \'ecoulement.
Le manom\`etre diff\'erentiel en U du tube de Pitot est rempli de mercure
de masse volumique $\rho_{Hg}$ = 13 600 kg.m$^{-3}$.
\begin{enumerate}
\item
Quel est le d\'ebit d'air obtenu dans l'hypoth\`ese d'un \'ecoulement
incompressible sachant que la d\'enivellation $\Delta h$ du mercure est
de 10 mm ?
\item
En consid\'erant que les effets de compressibilit\'e peuvent \^etre
n\'eglig\'ees pour des fluctuations de masse volumique de l'air $\rho$
inf\'erieures \`a 1\%,
jusqu'\`a quel nombre de Mach l'\'ecoulement peut \^etre consid\'er\'e
comme incompressible~?
Quelle est alors l'erreur commise sur l'\'evaluation de la vitesse ?
\end{enumerate}

%--------------------------------------------------------------------------------------------------
\subsection{Vol subsonique}
%--------------------------------------------------------------------------------------------------

Un avion vole \`a un nombre de Mach de 0.95 \`a une altitude o\`u la pression
atmosph\'erique est \'egale \`a 0.223 bar et o\`u la masse volumique de l'air
est $\rho=0.349$ kg.m$^{-3}$.
On suppose que l'air se comporte comme un fluide parfait et que les filets
fluides sont thermiquement isol\'es.
\begin{enumerate}
\item
Calculer la vitesse de l'avion en km/h.
\item
Calculer la pression et la temp\'erature au point d'arr\^et sur le bord
d'attaque de l'aile.
\end{enumerate}
 
%--------------------------------------------------------------------------------------------------
\subsection{Ecoulement isentropique}
%--------------------------------------------------------------------------------------------------

Dans un turbo-r\'eacteur, les gaz entrent dans la tuy\`ere \`a la vitesse
de 275 m.s$^{-1}$ et \`a une temp\'erature de 741 $^o$C.
Leur vitesse de sortie est de 564 m.s$^{-1}$.
En supposant le fluide parfait et l'\'ecoulement isentropique, calculer
\begin{enumerate}
\item
la temp\'erature de sortie des gaz,
\item
le nombre de Mach dans les conditions d'\'ejection des gaz.
\end{enumerate}
On assimilera les gaz \`a de l'air pur ($C_p = 1000$ J.kg$^{-1}$.K$^{-1}$).

%--------------------------------------------------------------------------------------------------
\subsection{R\'egimes isentropiques dans une tuyère}
%--------------------------------------------------------------------------------------------------

On consid\`ere dans ce probl\`eme l'\'ecoulement isentropique
d'un gaz parfait dans une tuy\`ere
intercal\'ee entre un r\'eservoir \`a air comprim\'e et l'atmosph\`ere
\`a pression $P_a = 1$ atm.
On assimilera le gaz \`a de l'air pur ($\gamma = 1.405$, 
$r=287$ J/kg/K).
La section de sortie a un diam\`etre $D_s = 5$ cm.

\begin{enumerate}
\setcounter{enumi}{0}
\item[]
\textbf{R\'egime 1 :}
\item[]
Dans ce r\'egime, la vitesse du gaz au niveau du col de la tuy\`ere
est $u_c = 250$ m/s.
La temp\'erature dans le r\'eservoir est $T_i = 300$ K.
\item
Calculer la temp\'erature au col $T_c$
puis le nombre de Mach au col $M_c$.
\item
En d\'eduire la nature (subsonique, supersonique ou sonique)
de l'\'ecoulement dans le convergent, au col et dans le divergent.
\item
\label{it:T_ext}
Par continuit\'e, la temp\'erature en sortie de la tuy\`ere est \'egale
\`a la temp\'erature ext\'erieure ambiante $T_s = T_a = 296$ K.
Calculer le nombre de Mach $M_s$ dans la section de sortie.
\item
\label{it:P_ext}
De la m\^eme fa\c{c}on, la pression en sortie de tuy\`ere $P_s$ correspond
\`a la pression atmosph\'erique $P_a$.
En d\'eduire la pression g\'en\'eratrice $P_i$ dans le r\'eservoir
puis la masse volumique g\'en\'eratrice $\rho_i$.
\item
Calculer la vitesse en sortie $u_s$ et la masse volumique en sortie $\rho_s$
puis en d\'eduire le d\'ebit massique $\dot{m}$ de la tuy\`ere.
\item[]
\textbf{R\'egime 2 :}
\item[]
Dans cette partie, la pression $P_i$ dans le r\'eservoir est augment\'ee
jusqu'\`a ce que le nombre de Mach en sortie atteigne $M_s = 1.5$.
Pour ce r\'egime, on observe ni onde de choc ni onde de d\'etente~:
la tuy\`ere est dite \textit{adapt\'ee}.
\item
D\'eterminer la nature (subsonique, supersonique ou sonique)
de l'\'ecoulement dans le convergent, au col et dans le divergent.
\item
Les conditions thermodynamiques en sortie de tuy\`ere sont les m\^emes
que dans les questions \ref{it:T_ext} et \ref{it:P_ext}.
En d\'eduire la pression g\'en\'eratrice $P_i$ et la temp\'erature
g\'en\'eratrice $T_i$ r\'egnant dans le r\'eservoir \`a air comprim\'e.
\item
	Calculer la temp\'erature au col $T_c$.
\item
	Calculer le diam\`etre de la section au col $D_c$.
\item
	Calculer le d\'ebit massique $\dot{m}$ pour ce r\'egime de fonctionnement de la tuy\`ere.
\end{enumerate}

%==================================================================================================
%\section{Ondes de choc}
%==================================================================================================

\setcounter{section}{11}

\section{Ondes de choc}


%--------------------------------------------------------------------------------------------------
\subsection{Tube de Pitot en \'ecoulement supersonique}
%--------------------------------------------------------------------------------------------------

On considère un \'ecoulement supersonique uniforme dont on cherche à mesurer les caract\'eristiques
(vitesse $u_1$, nombre de Mach $M_1$, pression $p_1$, pression g\'en\'eratrice $p_{i1}$) 
à l'aide d'un tube de Pitot double.
\begin{enumerate}
\item
D\'ecrire qualitativement l'\'ecoulement dans le voisinage du tube de Pitot.
\item
D\'eterminer l'expression du nombre de Mach $M_2$ au niveau du tube de Pitot en fonction
des pressions $p_2$ et $p_{i2}$ mesur\'ees par le tube de Pitot.
\item
En d\'eduire $M_1$.
\item
D\'eterminer $p_1$ et $p_{i1}$.
\item
Expliquer comment il serait possible de pr\'edire la vitesse $u_1$ de l'\'ecoulement incident. 
\end{enumerate}
 
%--------------------------------------------------------------------------------------------------
\subsection{R\'egime d'\'ecoulement discontinu dans une tuyère}
%--------------------------------------------------------------------------------------------------

On considère un \'ecoulement dans une tuyère caract\'eris\'e par la pr\'esence d'une onde de choc.
On souhaite d\'ecrire ce r\'egime particulier de fonctionnement de la tuyère.   
\begin{enumerate}
\item
Pr\'eciser dans quelle partie de la tuyère se trouve cette onde de choc.
Peut-il exister une autre onde de choc dans la tuyère ?
\item
Donner le nombre de Mach au col $M_c$.
\item
Tracer pour un tel r\'egime l'allure de la vitesse, de la pression et du nombre de Mach
le long de la tuyère.
\item
Que devient l'onde de choc lorsque la pression en sortie $p_s$ diminue ?
\item 
D\'eterminer en fonction des conditions g\'en\'eratrices en amont la plage
de pression $p_s$ correspondant à ce r\'egime d'\'ecoulement discontinu dans la tuyère.
\end{enumerate}
 
%--------------------------------------------------------------------------------------------------
\subsection{Onde de choc dans une tuyère}
%--------------------------------------------------------------------------------------------------

Une tuyère convergente-divergente est aliment\'ee avec de l'air à pression
g\'en\'eratrice $p_i$ de 3 bars et une temp\'erature $T_i$ de 600 K.
Cette tuyère a une section d'entr\'ee $A_e$ de 5 cm$^2$ dans laquelle la vitesse
$u_e$ a pour valeur 146 m.s$^{-1}$.
On constate lors du fonctionnement la pr\'esence d'une onde de choc dans
la section $A\indice{choc}$ d'aire 2.53 cm$^2$.
L'objectif de cet exercice est d'utiliser les relations isentropiques et de saut afin
de d\'eterminer certaines caract\'eristiques de l'\'ecoulement \'etudi\'e.
\begin{enumerate}
\item
  Calculer la pression $p_c$, la temp\'erature $T_c$ et la masse volumique $\rho_c$ au col ainsi
  que le d\'ebit massique $\dot{m}$ de la tuyère.
\item
  Calculer les pressions $p_1$ et $p_2$ et les temp\'eratures $T_1$ et $T_2$
  imm\'ediatement en amont et en aval de l'onde de choc.
\item
  Calculer la pression $p_s$ et la temp\'erature $T_s$ dans la section de sortie $A_s$ d'aire
  2.70 cm$^2$.
\end{enumerate}

 

