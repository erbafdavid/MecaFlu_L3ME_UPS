%%%%%%%%%%%%%%%%%%%%%%%%%%%%%%%%%%%%%%%%%%%%%%%%%%%%%%%%%%%%%%%%%%%%%%%%%%%%%%%%%%%%%%%%%%%%%%%%%%%
\documentclass[10pt, a4paper]{article}
%%%%%%%%%%%%%%%%%%%%%%%%%%%%%%%%%%%%%%%%%%%%%%%%%%%%%%%%%%%%%%%%%%%%%%%%%%%%%%%%%%%%%%%%%%%%%%%%%%%

%--------------------------------------------------------------------------------------------------
% Dimensions :
%--------------------------------------------------------------------------------------------------

\setlength{\textheight}{25cm}
\setlength{\textwidth}{16cm}

\setlength{\topmargin}{-20mm}
\setlength{\oddsidemargin}{0mm}
\setlength{\evensidemargin}{0mm}

% \setlength{\columnsep}{20mm}

\setlength{\fboxsep}{1mm}
\setlength{\unitlength}{1mm}

\usepackage{latexsym}
\usepackage{graphicx}
\usepackage{pifont}
\usepackage{color}
\usepackage{amsmath}
\usepackage{amssymb}

\usepackage[french]{babel}    % pour franciser le document

%\usepackage[latin1]{inputenc} % pour utiliser les caracteres accentues du claviers
\usepackage[utf8]{inputenc}

%--------------------------------------------------------------------------------------------------
% Divers :
%--------------------------------------------------------------------------------------------------

\definecolor{rougefonce}{rgb}{0.7, 0.2, 0.2}

\usepackage{accents}


% Style des vecteurs : fleche ou gras ?

%\newcommand{\myvec}[1]{\boldsymbol{#1}}
\newcommand{\myvec}[1]{\vec{#1}}

\newcommand{\mytensor}[1]{\accentset{\Rightarrow}{#1}} % needs \usepackage{accents}

%---------------------------
% Operateurs differentiels :
%---------------------------

\newcommand{\divergence}{\mbox{\rm div}\,}

\newcommand{\gradient}{\myvec{\mbox{\rm gra}}\mbox{\rm d}}
% \newcommand{\gradient}{\mathbf{grad}\,}
% \newcommand{\ggradient}{\stackrel{\Rightarrow}{\mbox{gra}}\!\!\!\,\mbox{d}\,}
\newcommand{\ggradient}{\accentset{\Rightarrow}{\mbox{\rm gra}}\mbox{\rm d}\!}

%\renewcommand{\dot}[1]{\accentset{\hbox{\huge .}}{#1}}
\newcommand{\mydot}[1]{\accentset{\centerdot}{#1}}

\newcommand{\rot}{\vec{\mbox{\rm ro}}\mbox{\rm t}\,}
%\newcommand{\rot}{\mathbf{rot}\,}

% \newcommand{\vnabla}{\vec{\nabla}}
\newcommand{\vnabla}{\boldsymbol{\nabla}}

% Fonctions speciales:

\newcommand{\besselj}[1]{\mbox{J}_{#1}}
\newcommand{\besselk}[1]{\mbox{K}_{#1}}
\newcommand{\bessely}[1]{\mbox{Y}_{#1}}
\newcommand{\besseli}[1]{\mbox{I}_{#1}}

% Vecteurs, tenseurs et torseurs:

\newcommand{\ex}{\myvec{e}_{x}}
\newcommand{\ey}{\myvec{e}_{y}}
\newcommand{\ez}{\myvec{e}_{z}}

\newcommand{\er}{\myvec{e}_{r}}
\newcommand{\erho}{\myvec{e}_{\rho}}
\newcommand{\ephi}{\myvec{e}_{\varphi}}
\newcommand{\etheta}{\myvec{e}_{\theta}}

%\newcommand{\tensor}[1]{\stackrel{\Rightarrow}{#1}}
\newcommand{\tensor}[1]{\mbox{\sl \textbf{#1}}}
\newcommand{\torseur}[4]{
   \!\!\!\! \left . \begin{array}{c} \\ \\ _#1 \end{array} \!\!\!
   \right \{ \!\!\!
   \begin{array}{#4} #2 \\ \\ #3 \end{array}}

% Integrales multiples:

\newcommand{\odblint}[1]{\int\!\!\!\!\!\int_{#1} \hskip -7mm \bigcirc \;}
\newcommand{\dblint}{\int\!\!\!\!\!\int}
\newcommand{\tplint}{\int\!\!\!\!\!\int\!\!\!\!\!\int}

% Fractions:

\renewcommand{\dfrac}[2]{\displaystyle \frac{#1}{#2}}

% Derivees ordinaires et partielles:

\newcommand{\dpdt}[1]{\dfrac{\partial #1}{\partial t}}
\newcommand{\dpdx}[1]{\dfrac{\partial #1}{\partial x}}
\newcommand{\dpdy}[1]{\dfrac{\partial #1}{\partial y}}
\newcommand{\dpdz}[1]{\dfrac{\partial #1}{\partial z}}

\newcommand{\ddpdt}[1]{\dfrac{\partial^2 #1}{\partial t^2}}
\newcommand{\ddpdx}[1]{\dfrac{\partial^2 #1}{\partial x^2}}
\newcommand{\ddpdy}[1]{\dfrac{\partial^2 #1}{\partial y^2}}
\newcommand{\ddpdz}[1]{\dfrac{\partial^2 #1}{\partial z^2}}

\newcommand{\dpdr}[1]{\dfrac{\partial #1}{\partial r}}
\newcommand{\dpdrho}[1]{\dfrac{\partial #1}{\partial \rho}}
\newcommand{\dpdphi}[1]{\dfrac{\partial #1}{\partial \varphi}}
\newcommand{\dpdtheta}[1]{\dfrac{\partial #1}{\partial \theta}}

\newcommand{\ddpdr}[1]{\dfrac{\partial^2 #1}{\partial r^2}}
\newcommand{\ddpdrho}[1]{\dfrac{\partial^2 #1}{\partial \rho^2}}
\newcommand{\ddpdphi}[1]{\dfrac{\partial^2 #1}{\partial \varphi^2}}
\newcommand{\ddpdtheta}[1]{\dfrac{\partial ^2#1}{\partial \theta^2}}

\newcommand{\ddt}[1]{\dfrac{d #1}{dt}}
\newcommand{\ddx}[1]{\dfrac{d #1}{dx}}
\newcommand{\ddy}[1]{\dfrac{d #1}{dy}}
\newcommand{\ddz}[1]{\dfrac{d #1}{dz}}
\newcommand{\ddr}[1]{\dfrac{d #1}{dr}}

\newcommand{\ddtref}[2]{\dfrac{d #1}{dt}_{\! | #2 }}
\newcommand{\dpdtref}[3]{\dfrac{\partial #1}{\partial #2}_{\! | #3 }}

% Misc:

\newcommand{\bareme}[1]{\reversemarginpar \marginpar{%
	\hspace{8mm} \small \makebox[5mm]{/#1}% 
	\begin{picture}(0, 0)\put(0, 0.5){\makebox[12mm]{\dotfill}}\end{picture}%
	}}

\newcommand{\mycaption}[1]{\caption{\sl #1}}

\newcommand{\ligne}[1]{\hrule height #1\linethickness \hfill}

\newcommand{\thickline}[2]{\linethickness{#1} \line(1, 0){#2}}

\newcommand{\myline}{\noindent\underline{\hspace{\textwidth}}}
\newcommand{\mysection}[1]{\vskip 0.5cm \section{#1}\vskip -1.4cm 
   \myline \vskip 0.4cm \myline \bigskip}

\newcommand{\etal}{\textit{et al.}}

\newcommand{\varray}[1]{\renewcommand{\arraystretch}{#1}}

\newcommand{\puissance}[1]{^{\mbox{\footnotesize #1}}}
\newcommand{\indice}[1]{_{\mbox{\footnotesize #1}}}

%---------------------------------------------------------------------
% New environments:
%---------------------------------------------------------------------

\newcounter{MyEnumCounter}
\newcounter{MySaveCounter}
\newenvironment{MyEnum}{%
  \begin{list}{\arabic{MyEnumCounter}.}{\usecounter{MyEnumCounter}%
  \setcounter{MyEnumCounter}{\value{MySaveCounter}}}
  }{%
  \setcounter{MySaveCounter}{\value{MyEnumCounter}}\end{list}%
}
\newcommand{\MyEnumReset}{\setcounter{MySaveCounter}{0}}

\newenvironment{deuxcols}{\begin{tabular}{lr} \hspace*{-9.7mm}}{\end{tabular}}

\newenvironment{dem}{\noindent %
   \begin{tabular}{||l} \textsl{D\'emonstration :} \\ % 
   \begin{minipage}{15.5cm} \footnotesize} %
   {\end{minipage}\end{tabular}}

\newenvironment{abst}{\begin{quotation}\sl}{\end{quotation}}

\newenvironment{eqnbox}{\begin{equation}\begin{array}{|c|}  \hline \\ 
   \displaystyle}{\\ \\ \hline \end{array} \end{equation}}

% JFM symbols:

\DeclareMathSymbol{\varGamma}{\mathord}{letters}{"00}
\DeclareMathSymbol{\varDelta}{\mathord}{letters}{"01}
\DeclareMathSymbol{\varTheta}{\mathord}{letters}{"02}
\DeclareMathSymbol{\varLambda}{\mathord}{letters}{"03}
\DeclareMathSymbol{\varXi}{\mathord}{letters}{"04}
\DeclareMathSymbol{\varPi}{\mathord}{letters}{"05}
\DeclareMathSymbol{\varSigma}{\mathord}{letters}{"06}
\DeclareMathSymbol{\varUpsilon}{\mathord}{letters}{"07}
\DeclareMathSymbol{\varPhi}{\mathord}{letters}{"08}
\DeclareMathSymbol{\varPsi}{\mathord}{letters}{"09}
\DeclareMathSymbol{\varOmega}{\mathord}{letters}{"0A}

% ---------------------------------------------------------------------
% MISC SYMBOLS :
% ---------------------------------------------------------------------

\font\SY=msam10 
\def\carreblanc{\hbox{\SY \char'3}}
\def\carrenoir{\hbox{\SY \char'4}}
\def\diamblanc{\hbox{\SY \char'6}}
\def\diamnoir{\hbox{\SY \char'7}}
\def\triblancright{\hbox{\SY \char'102}}
\def\triblancleft{\hbox{\SY \char'103}}
\def\triblancup{\hbox{\SY \char'115}}
\def\triblancdown{\hbox{\SY \char'117}}
\def\trinoirright{\hbox{\SY \char'111}}
\def\trinoirleft{\hbox{\SY \char'112}}
\def\trinoirup{\hbox{\SY \char'116}}
\def\trinoirdown{\hbox{\SY \char'110}}
\def\rondblanc{\hbox{\scriptsize $\bigcirc$}}
\def\rondnoir{\hbox{\LARGE $\bullet$}}

\font\BB=msbm10 scaled 1095
\def\setr{\hbox{\BB R}}
\def\setc{\hbox{\BB C}}
\def\setn{\hbox{\BB N}}
\def\setz{\hbox{\BB Z}}

% Pour enlever la numerotation des pages de la table des matieres:

%%%% debut macro, a placer dans preambule %%%%
\makeatletter
\def\addcontentsline@toc#1#2#3{%
   \addtocontents{#1}{\protect\thispagestyle{empty}}%
   \addtocontents{#1}{\protect\contentsline{#2}{#3}{\thepage}}}
\def\addcontentsline#1#2#3{%
  \expandafter\@ifundefined{addcontentsline@#1}%
  {\addtocontents{#1}{\protect\contentsline{#2}{#3}{\thepage}}}
  {\csname addcontentsline@#1\endcsname{#1}{#2}{#3}}}
\makeatother
%%%% fin macro %%%%

\newcommand{\titre}[1]{ %
  \medskip \noindent \underline{\makebox[\textwidth][l]{\textbf{#1}\textcolor{white}{pl}}}}% \\}

\newcommand{\sstitre}[1]{ %
  \bigskip \centerline{\textbf{#1}} \smallskip}

\def\draft{\overfullrule 5pt} % The \draft command marks the overful boxes

\def\indentlist{\list%
        {}{\labelwidth 0pt \leftmargin 3\labelsep}}
\let\endindentlist\endlist \relax

\def\datelist{\list%
        {}{\settowidth\labelwidth{[2001/02 :]}
        \leftmargin\labelwidth
        \advance\leftmargin\labelsep}
}
\let\enddatelist\endlist \relax

\def\longuelist{\list%
        {}{\settowidth\labelwidth{[Etablissement :]}
        \leftmargin\labelwidth
        \advance\leftmargin\labelsep}
}
\let\endlonguelist\endlist \relax

\def\shortlist{\list%
        {}{\settowidth\labelwidth{$\bullet$}
        \leftmargin\labelwidth
        \advance\leftmargin\labelsep}
}
\let\endshortlist\endlist \relax



\renewcommand{\thickline}[2]{\linethickness{#1} \line(1, 0){#2}}
\renewcommand{\mycaption}[1]{\caption{\sl #1}}
\renewcommand{\myvec}[1]{\vec{#1}}


% \pagestyle{empty}

\graphicspath{{Figures/}} % chemin d'acces au repertoire des figures (par ex.)

%%%%%%%%%%%%%%%%%%%%%%%%%%%%%%%%%%%%%%%%%%%%%%%%%%%%%%%%%%%%%%%%%%%%%%%%%%%%%%%%%%%%%%%%%%%%%%%%%%%
\begin{document}
%%%%%%%%%%%%%%%%%%%%%%%%%%%%%%%%%%%%%%%%%%%%%%%%%%%%%%%%%%%%%%%%%%%%%%%%%%%%%%%%%%%%%%%%%%%%%%%%%%%

\begin{center}

  \textsc{Université Toulouse 3 -- Paul Sabatier \hfill Année universitaire 2017-2018}
  
  \textsc{Mécanique des fluides \hfill L3 Mécanique}
  
  \vspace{0mm}
  
  \begin{center}
    \thickline{0.4mm}{160}
    \\ \vspace{3mm}
  \textbf{\large Ecoulements linéaires (exercice préparatoire chapitre cinématique)}
    \\ %\vspace{1mm}
    \thickline{0.4mm}{160}
  \end{center}

%  \vspace{0mm}
  
\end{center}

%==================================================================================================
\section{Introduction}
%==================================================================================================

\paragraph{Définition :\\}

Considérons un écoulement plan $\myvec{u} = u(x, y, t) \, \ex + v(x, y, t) \, \ey$ 
%et incompressible (ou isovolume), c'est-à-dire
Défini par :
\begin{eqnarray}
u(x,y) &=& a x + by\\
v(x,y) &=& cx+dy
\end{eqnarray} 


\paragraph{Propriétés :}

\begin{enumerate}
\item
	A quelle condition cet écoulement est-il a divergence nulle ? (on dit aussi écoulement isovolume, ou par abus de language écoulement incompressible).
\item
	Justifiez que dans ce cas on peut introduire une fonction de courant $\psi(x,y)$, et donnez son expression.
	
	\item
	Montrer que les lignes de courant de l'écoulement sont les courbes le long desquelles 
	la fonction de courant est constante, définies par $\psi(x, y) = Cte$.
\item
	Question subsidiaire : montrer que le débit est conservé entre deux lignes de courants.
	Plus préci\-sé\-ment, si l'on considère deux lignes de courant  définies 
	par $\psi(x, y) = C_1$ et $\psi(x, y) = C_2$,
	alors le débit qui passe entre ces deux lignes de courant est donné par $q=C_2-C_1$.
\end{enumerate}

%==================================================================================================
\section{Ecoulements linéaires}
%==================================================================================================

On considère dans cet exercice les écoulements plans stationnaires définis par la fonction de courant 
\begin{equation}
	\psi(x, y) = -c x^{2}/2 + b y^{2}/2
	\label{eq:linearflow}
\end{equation}
où $b$ et $c$ sont deux paramètres constants qui définissent la nature de l'écoulement considéré.

\begin{enumerate}
\item
	Calculer en fonction de $b$ et $c$ le champ de vitesse correspondant à la fonction de courant donnée par l'expression (\ref{eq:linearflow}).
\item
	Vérifier à nouveau qu'il s'agit bien d'un écoulement incompressible.
\item
	Déterminer la dérivée particulaire du champ de vitesse obtenu.
	Comment est modifiée cette dérivée particulaire si on considère des coefficients dépendants
	du temps $c(t)$ et $b(t)$ ?
\item
	Donner l'expression du tenseur des gradients de vitesse $\mytensor{G}$. 
	On justifiera qu'on pourra se restreindre à un tenseur représenté par une matrice $2\times2$.
\item
	Justifier alors l'appellation d'écoulement \textsl{linéaire} associée à cette famille 
	particulière de fonction de courant.
\item
	Déterminer le tenseur des taux de déformation $\mytensor{D}$. A quelle condition ce tenseur est nul ?
\item
	Déterminer le tenseur des taux de rotation $\mytensor{R}$. A quelle condition ce tenseur est nul ?
\item
	Déduire des deux dernières questions une décomposition canonique de la fonction de courant $\psi$.
\item
	Calculer le vecteur tourbillon $\myvec{\varOmega}$
	et le champ de vorticité $\myvec{\omega}$ de l'écoulement.

\item On effectue maintenant un changement de repère en faisant tourner les axes d'un angle $\theta$. On pose donc  : 

$x' = x \cos \theta + y \sin \theta$ ;  $y' = -x \sin \theta + y \cos \theta$ ;
$u_x' = u_x \cos \theta + u_y \sin \theta$ ;  $u_y' = -u_x \sin \theta + u_y \cos \theta$.

Calculez le tenseur du gradient des vitesses, ainsi que ses composantes symétrique et antisymétrique, dans ces nouveaux axes  
(on notera $\mytensor{G}_{(x'y')} =\mytensor{D}_{(x'y')} + \mytensor{R}_{(x'y')}$).

\item Que remarque-t-on pour la partie antisymétrique (tenseur des taux de rotation) ?

\item Pour la partie symétrique (tenseur des taux de déformation), montrez qu'il existe un changement de variable rendant le tenseur diagonal. Interprétez.


\end{enumerate}

%==================================================================================================
\section{Ecoulements linéaires : cas particuliers}
%==================================================================================================

\begin{enumerate}

\item
	Ecrire un programme Matlab qui trace les lignes de courant 
	pour n'importe quelle valeur de $c$ et $b$ choisies par l'utilisateur.

\item
	On considère le cas particulier $c=0$.
	\begin{enumerate}
	\item 
  		déterminer les lignes de courant et les trajectoires (on vérifiera qu'il s'agit 
  		des mêmes courbes),
	\item 
  		déterminer l'accélération des particules fluides.
	\item 
  		commenter la forme des tenseurs des taux de déformation et des taux de rotation dans ce cas.
	\end{enumerate}
\item
	Mêmes questions dans le cas $c/b=1$ (on pourra éventuellement prendre $b=1$ sans perte de généralité).
	Tracer les lignes de courant avec Matlab et l'accélération des particules fluides en quelques points particuliers.
\item
	Mêmes questions dans le cas $c/b=-1$.
\item
	Question subsidiaire : reprendre le programme Matlab précédent afin qu'il
	calcule l'évolution d'un élément matériel carré initialement centré en un point $(x_0, y_0)$ 
	et de côté $L$ choisis par l'utilisateur.	
	
\end{enumerate}


%%%%%%%%%%%%%%%%%%%%%%%%%%%%%%%%%%%%%%%%%%%%%%%%%%%%%%%%%%%%%%%%%%%%%%%%%%%%%%%%%%%%%%%%%%%%%%%%%%%
\end{document}
%%%%%%%%%%%%%%%%%%%%%%%%%%%%%%%%%%%%%%%%%%%%%%%%%%%%%%%%%%%%%%%%%%%%%%%%%%%%%%%%%%%%%%%%%%%%%%%%%%%

