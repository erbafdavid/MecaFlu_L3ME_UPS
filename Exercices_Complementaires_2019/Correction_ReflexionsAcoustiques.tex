%%%%%%%%%%%%%%%%%%%%%%%%%%%%%%%%%%%%%%%%%%%%%%%%%%%%%%%%%%%%%%%%%%%%%%%%%%%%%%%%%%%%%%%%%%%%%%%%%%%
\documentclass[10pt, a4paper]{article}
%%%%%%%%%%%%%%%%%%%%%%%%%%%%%%%%%%%%%%%%%%%%%%%%%%%%%%%%%%%%%%%%%%%%%%%%%%%%%%%%%%%%%%%%%%%%%%%%%%%

%--------------------------------------------------------------------------------------------------
% Dimensions :
%--------------------------------------------------------------------------------------------------

\setlength{\textheight}{26cm}
\setlength{\textwidth}{16cm}

\setlength{\topmargin}{-25mm}
\setlength{\oddsidemargin}{0mm}
\setlength{\evensidemargin}{0mm}

% \setlength{\columnsep}{20mm}

\setlength{\fboxsep}{1mm}
\setlength{\unitlength}{1mm}

%--------------------------------------------------------------------------------------------------
% Packages :
%--------------------------------------------------------------------------------------------------

\usepackage{latexsym}
\usepackage{graphicx}
\usepackage{pifont}
\usepackage{color}
\usepackage{amsmath}
\usepackage{amssymb}
\usepackage{enumerate}

\usepackage[french]{babel}    % pour franciser le document

%\usepackage[latin1]{inputenc} % pour utiliser les caracteres accentues du claviers
\usepackage[utf8]{inputenc} 

\usepackage{accents}


% Style des vecteurs : fleche ou gras ?

%\newcommand{\myvec}[1]{\boldsymbol{#1}}
\newcommand{\myvec}[1]{\vec{#1}}

\newcommand{\mytensor}[1]{\accentset{\Rightarrow}{#1}} % needs \usepackage{accents}

%---------------------------
% Operateurs differentiels :
%---------------------------

\newcommand{\divergence}{\mbox{\rm div}\,}

\newcommand{\gradient}{\myvec{\mbox{\rm gra}}\mbox{\rm d}}
% \newcommand{\gradient}{\mathbf{grad}\,}
% \newcommand{\ggradient}{\stackrel{\Rightarrow}{\mbox{gra}}\!\!\!\,\mbox{d}\,}
\newcommand{\ggradient}{\accentset{\Rightarrow}{\mbox{\rm gra}}\mbox{\rm d}\!}

%\renewcommand{\dot}[1]{\accentset{\hbox{\huge .}}{#1}}
\newcommand{\mydot}[1]{\accentset{\centerdot}{#1}}

\newcommand{\rot}{\vec{\mbox{\rm ro}}\mbox{\rm t}\,}
%\newcommand{\rot}{\mathbf{rot}\,}

% \newcommand{\vnabla}{\vec{\nabla}}
\newcommand{\vnabla}{\boldsymbol{\nabla}}

% Fonctions speciales:

\newcommand{\besselj}[1]{\mbox{J}_{#1}}
\newcommand{\besselk}[1]{\mbox{K}_{#1}}
\newcommand{\bessely}[1]{\mbox{Y}_{#1}}
\newcommand{\besseli}[1]{\mbox{I}_{#1}}

% Vecteurs, tenseurs et torseurs:

\newcommand{\ex}{\myvec{e}_{x}}
\newcommand{\ey}{\myvec{e}_{y}}
\newcommand{\ez}{\myvec{e}_{z}}

\newcommand{\er}{\myvec{e}_{r}}
\newcommand{\erho}{\myvec{e}_{\rho}}
\newcommand{\ephi}{\myvec{e}_{\varphi}}
\newcommand{\etheta}{\myvec{e}_{\theta}}

%\newcommand{\tensor}[1]{\stackrel{\Rightarrow}{#1}}
\newcommand{\tensor}[1]{\mbox{\sl \textbf{#1}}}
\newcommand{\torseur}[4]{
   \!\!\!\! \left . \begin{array}{c} \\ \\ _#1 \end{array} \!\!\!
   \right \{ \!\!\!
   \begin{array}{#4} #2 \\ \\ #3 \end{array}}

% Integrales multiples:

\newcommand{\odblint}[1]{\int\!\!\!\!\!\int_{#1} \hskip -7mm \bigcirc \;}
\newcommand{\dblint}{\int\!\!\!\!\!\int}
\newcommand{\tplint}{\int\!\!\!\!\!\int\!\!\!\!\!\int}

% Fractions:

\renewcommand{\dfrac}[2]{\displaystyle \frac{#1}{#2}}

% Derivees ordinaires et partielles:

\newcommand{\dpdt}[1]{\dfrac{\partial #1}{\partial t}}
\newcommand{\dpdx}[1]{\dfrac{\partial #1}{\partial x}}
\newcommand{\dpdy}[1]{\dfrac{\partial #1}{\partial y}}
\newcommand{\dpdz}[1]{\dfrac{\partial #1}{\partial z}}

\newcommand{\ddpdt}[1]{\dfrac{\partial^2 #1}{\partial t^2}}
\newcommand{\ddpdx}[1]{\dfrac{\partial^2 #1}{\partial x^2}}
\newcommand{\ddpdy}[1]{\dfrac{\partial^2 #1}{\partial y^2}}
\newcommand{\ddpdz}[1]{\dfrac{\partial^2 #1}{\partial z^2}}

\newcommand{\dpdr}[1]{\dfrac{\partial #1}{\partial r}}
\newcommand{\dpdrho}[1]{\dfrac{\partial #1}{\partial \rho}}
\newcommand{\dpdphi}[1]{\dfrac{\partial #1}{\partial \varphi}}
\newcommand{\dpdtheta}[1]{\dfrac{\partial #1}{\partial \theta}}

\newcommand{\ddpdr}[1]{\dfrac{\partial^2 #1}{\partial r^2}}
\newcommand{\ddpdrho}[1]{\dfrac{\partial^2 #1}{\partial \rho^2}}
\newcommand{\ddpdphi}[1]{\dfrac{\partial^2 #1}{\partial \varphi^2}}
\newcommand{\ddpdtheta}[1]{\dfrac{\partial ^2#1}{\partial \theta^2}}

\newcommand{\ddt}[1]{\dfrac{d #1}{dt}}
\newcommand{\ddx}[1]{\dfrac{d #1}{dx}}
\newcommand{\ddy}[1]{\dfrac{d #1}{dy}}
\newcommand{\ddz}[1]{\dfrac{d #1}{dz}}
\newcommand{\ddr}[1]{\dfrac{d #1}{dr}}

\newcommand{\ddtref}[2]{\dfrac{d #1}{dt}_{\! | #2 }}
\newcommand{\dpdtref}[3]{\dfrac{\partial #1}{\partial #2}_{\! | #3 }}

% Misc:

\newcommand{\bareme}[1]{\reversemarginpar \marginpar{%
	\hspace{8mm} \small \makebox[5mm]{/#1}% 
	\begin{picture}(0, 0)\put(0, 0.5){\makebox[12mm]{\dotfill}}\end{picture}%
	}}

\newcommand{\mycaption}[1]{\caption{\sl #1}}

\newcommand{\ligne}[1]{\hrule height #1\linethickness \hfill}

\newcommand{\thickline}[2]{\linethickness{#1} \line(1, 0){#2}}

\newcommand{\myline}{\noindent\underline{\hspace{\textwidth}}}
\newcommand{\mysection}[1]{\vskip 0.5cm \section{#1}\vskip -1.4cm 
   \myline \vskip 0.4cm \myline \bigskip}

\newcommand{\etal}{\textit{et al.}}

\newcommand{\varray}[1]{\renewcommand{\arraystretch}{#1}}

\newcommand{\puissance}[1]{^{\mbox{\footnotesize #1}}}
\newcommand{\indice}[1]{_{\mbox{\footnotesize #1}}}

%---------------------------------------------------------------------
% New environments:
%---------------------------------------------------------------------

\newcounter{MyEnumCounter}
\newcounter{MySaveCounter}
\newenvironment{MyEnum}{%
  \begin{list}{\arabic{MyEnumCounter}.}{\usecounter{MyEnumCounter}%
  \setcounter{MyEnumCounter}{\value{MySaveCounter}}}
  }{%
  \setcounter{MySaveCounter}{\value{MyEnumCounter}}\end{list}%
}
\newcommand{\MyEnumReset}{\setcounter{MySaveCounter}{0}}

\newenvironment{deuxcols}{\begin{tabular}{lr} \hspace*{-9.7mm}}{\end{tabular}}

\newenvironment{dem}{\noindent %
   \begin{tabular}{||l} \textsl{D\'emonstration :} \\ % 
   \begin{minipage}{15.5cm} \footnotesize} %
   {\end{minipage}\end{tabular}}

\newenvironment{abst}{\begin{quotation}\sl}{\end{quotation}}

\newenvironment{eqnbox}{\begin{equation}\begin{array}{|c|}  \hline \\ 
   \displaystyle}{\\ \\ \hline \end{array} \end{equation}}

% JFM symbols:

\DeclareMathSymbol{\varGamma}{\mathord}{letters}{"00}
\DeclareMathSymbol{\varDelta}{\mathord}{letters}{"01}
\DeclareMathSymbol{\varTheta}{\mathord}{letters}{"02}
\DeclareMathSymbol{\varLambda}{\mathord}{letters}{"03}
\DeclareMathSymbol{\varXi}{\mathord}{letters}{"04}
\DeclareMathSymbol{\varPi}{\mathord}{letters}{"05}
\DeclareMathSymbol{\varSigma}{\mathord}{letters}{"06}
\DeclareMathSymbol{\varUpsilon}{\mathord}{letters}{"07}
\DeclareMathSymbol{\varPhi}{\mathord}{letters}{"08}
\DeclareMathSymbol{\varPsi}{\mathord}{letters}{"09}
\DeclareMathSymbol{\varOmega}{\mathord}{letters}{"0A}

% ---------------------------------------------------------------------
% MISC SYMBOLS :
% ---------------------------------------------------------------------

\font\SY=msam10 
\def\carreblanc{\hbox{\SY \char'3}}
\def\carrenoir{\hbox{\SY \char'4}}
\def\diamblanc{\hbox{\SY \char'6}}
\def\diamnoir{\hbox{\SY \char'7}}
\def\triblancright{\hbox{\SY \char'102}}
\def\triblancleft{\hbox{\SY \char'103}}
\def\triblancup{\hbox{\SY \char'115}}
\def\triblancdown{\hbox{\SY \char'117}}
\def\trinoirright{\hbox{\SY \char'111}}
\def\trinoirleft{\hbox{\SY \char'112}}
\def\trinoirup{\hbox{\SY \char'116}}
\def\trinoirdown{\hbox{\SY \char'110}}
\def\rondblanc{\hbox{\scriptsize $\bigcirc$}}
\def\rondnoir{\hbox{\LARGE $\bullet$}}

\font\BB=msbm10 scaled 1095
\def\setr{\hbox{\BB R}}
\def\setc{\hbox{\BB C}}
\def\setn{\hbox{\BB N}}
\def\setz{\hbox{\BB Z}}

% Pour enlever la numerotation des pages de la table des matieres:

%%%% debut macro, a placer dans preambule %%%%
\makeatletter
\def\addcontentsline@toc#1#2#3{%
   \addtocontents{#1}{\protect\thispagestyle{empty}}%
   \addtocontents{#1}{\protect\contentsline{#2}{#3}{\thepage}}}
\def\addcontentsline#1#2#3{%
  \expandafter\@ifundefined{addcontentsline@#1}%
  {\addtocontents{#1}{\protect\contentsline{#2}{#3}{\thepage}}}
  {\csname addcontentsline@#1\endcsname{#1}{#2}{#3}}}
\makeatother
%%%% fin macro %%%%

\newcommand{\titre}[1]{ %
  \medskip \noindent \underline{\makebox[\textwidth][l]{\textbf{#1}\textcolor{white}{pl}}}}% \\}

\newcommand{\sstitre}[1]{ %
  \bigskip \centerline{\textbf{#1}} \smallskip}

\def\draft{\overfullrule 5pt} % The \draft command marks the overful boxes

\def\indentlist{\list%
        {}{\labelwidth 0pt \leftmargin 3\labelsep}}
\let\endindentlist\endlist \relax

\def\datelist{\list%
        {}{\settowidth\labelwidth{[2001/02 :]}
        \leftmargin\labelwidth
        \advance\leftmargin\labelsep}
}
\let\enddatelist\endlist \relax

\def\longuelist{\list%
        {}{\settowidth\labelwidth{[Etablissement :]}
        \leftmargin\labelwidth
        \advance\leftmargin\labelsep}
}
\let\endlonguelist\endlist \relax

\def\shortlist{\list%
        {}{\settowidth\labelwidth{$\bullet$}
        \leftmargin\labelwidth
        \advance\leftmargin\labelsep}
}
\let\endshortlist\endlist \relax



%--------------------------------------------------------------------------------------------------
% Divers :
%--------------------------------------------------------------------------------------------------

\definecolor{rougefonce}{rgb}{0.7, 0.2, 0.2}

\renewcommand{\thickline}[2]{\linethickness{#1} \line(1, 0){#2}}
\renewcommand{\mycaption}[1]{\caption{\sl #1}}
\renewcommand{\myvec}[1]{\vec{#1}}

\newcommand{\footnoteremember}[2]{
\footnote{#2}
\newcounter{#1}
\setcounter{#1}{\value{footnote}}% \!\!\!
}
\newcommand{\footnoterecall}[1]{
\footnotemark[\value{#1}]
}

\newcommand{\question}[1]{}
\newcommand{\answer}[1]{#1}

% \pagestyle{empty}

\graphicspath{{../FIGURES/}{Figures/}} % chemin d'acces au repertoire des figures (par ex.)

%%%%%%%%%%%%%%%%%%%%%%%%%%%%%%%%%%%%%%%%%%%%%%%%%%%%%%%%%%%%%%%%%%%%%%%%%%%%%%%%%%%%%%%%%%%%%%%%%%%
\begin{document}
%%%%%%%%%%%%%%%%%%%%%%%%%%%%%%%%%%%%%%%%%%%%%%%%%%%%%%%%%%%%%%%%%%%%%%%%%%%%%%%%%%%%%%%%%%%%%%%%%%%

\begin{center}

  \textsc{Université Toulouse 3 -- Paul Sabatier \hfill Année universitaire 2017-2018}
  
  \textsc{Mécanique des fluides \hfill L3 Mécanique}
  
  \vspace{0mm}
  
  \begin{center}
    \thickline{0.4mm}{160}
    \\ \vspace{3mm}
  \textbf{\large Exercice complémentaire 10 : 
  \\
  Réflexion/transmission d'une onde acoustique;  \\
  Application au fonctionnement de l'oreille moyenne\\
  {\em Eléments de correction}
  }
    \\ %\vspace{1mm}
    \thickline{0.4mm}{160}
  \end{center}

  %\vspace{5mm}
  
\end{center}

\section{Description d'une onde acoustique monodimensionnelle}

 
\begin{enumerate}
\item 
\question{Rappelez les hypothèses de l'acoustique linéaire. Montrez que sous ces hypothèses, l'évolution de la pression est gouvernée par l'équation de Helmholtz :
$$
\frac{\partial^2 p'}{\partial t^2} = c_0^2 \nabla^2 p'
$$
Définir $c_0$ dans le cas général, puis dans le cas d'un gaz parfait.
}
\answer{(Voir notes de cours)}

\item \question{
On considère une solution d'onde plane progressive monochromatique se déplaçant dans la direction $+\vec{e}_x$, sous la  forme suivante :
$$
p'(x,t) = Re (A e^{i (k x - \omega t)})
$$
Représentez la forme d'une telle onde à différents instants. 

A quoi correspondent les quantités $A$, $k$ et $\omega$ ? Quelle relation relie ces deux dernières quantités ?}
\answer{
Cette forme d'onde s'écrit aussi, en posant $A= |A| e^{i \varphi_A}$:
$$
p'(x,t) =|A| \cos (k x - \omega t+ \varphi_A)) 
$$
$k$ est le nombre d'onde relié à la longeur d'onde par $k = 2\pi/\lambda$ ; $\omega$ est la pulsation reliée à la période par $\omega = 2\pi/T$;
$\varphi_A$ est la phase de l'onde qui peut être choisie arbitrairement à zéro (en définissant convenablement l'origine des temps). 

Les dérivées seconde en temps et en espace s'écrivent respectivement : 
$$
\frac{\partial^2 p'}{\partial t^2} =- \omega^2 |A| \cos (k x - \omega t+ \varphi_A))   ; 
\qquad  \nabla^2 p' =  \frac{\partial^2 p'}{\partial x^2} =-k^2 |A| \cos (k x - \omega t+ \varphi_A)).
$$
En reportant dans l'équation des ondes et en simplifiant on obtient :
$$
\omega^2 = c_0^2 k^2
$$ 
si $k>0$ et $\omega>0$ on a donc $\omega = c_0 k$. C'est la "relation de dispersion".

On peut remarquer en outre que cette forme d'onde se met sous la forme :
$$p'(x,t) =|A| \cos (k (x -c_0 t) + \varphi_A) = f(x-c_0 t)$$
C'est donc bien une onde progressive se propageant à la vitesse $+c_0$.
}


\item 
\question{A partir des équations du mouvement, donnez la loi de vitesse $u'(x,t)$ correspondant à une telle onde.}
\answer{
En reportant (par exemple) dans l'équation de la quantité de mouvement, on a :
$$
\frac{\partial u'}{\partial t} =- \frac{1}{\rho_0} \frac{\partial p'}{\partial x} = - \frac{k |A|}{\rho_0} \sin (k x - \omega t+ \varphi_A)) 
$$ 
En intégrant par rapport au temps on a :
$$
\partial u'  = + \frac{k |A|}{\rho_0 \omega } \cos (k x - \omega t+ \varphi_A) = \frac{p'}{\rho_0 c_0}.
$$ 
(la constante d'intégration est ici prise égale à zéro car la moyenne en temps de $u'$, qui est par définition une perturbation de vitesse par rapport à un état au repos, est nulle).

Remarque : on peut aussi procéder de même a partir de l'équation de la masse ce qui conduit à la même relation.
}

\item 
\question{Définir l'énergie acoustique volumique $e_{ac}$ et le vecteur intensité acoustique $\vec{I}$. }
\answer{D'après le cours :
$$
e_{ac} = \frac{|p'|}{2 \rho c_0^2} +  \frac{\rho_0 |u'|}{2} = 
\frac{|A|^2  \cos^2 (k x - \omega t+ \varphi_A)) }{2 \rho_0 c_0^2}  
+ \frac{|A|^2  \cos^2 (k x - \omega t+ \varphi_A) }{2 \rho_0 c_0^2}
= 
\frac{|A|^2  \cos^2 (k x - \omega t+ \varphi_A) }{\rho_0 c_0^2} ; 
$$
$$
\vec{I} = p' \vec{u}' =  p' u'  \vec{e}_x = 
\frac{|A|^2  \cos^2 (k x - \omega t+ \varphi_A) }{\rho_0 c_0} \vec e_x  = c_0 e_{ac}  \vec{e}_x.
$$
}



\item \question{Montrez que dans le cas considéré ici, la valeur moyenne de ces quantités (au sens de la moyenne temporelle sur une période d'oscillation) sont données respectivement par :
 
$$
\overline{e_{ac}} = \frac{|A|^2}{2 \rho_0 c_0 ^2} ; \qquad \overline{I_x} = c_0  \overline{e_{ac}} 
$$
}
\answer{ En notant que $\overline{\cos^2 (k x - \omega t+ \varphi_A)} = 1/2$, on obtient directement le résultat demandé :
$$
\overline{e_{ac}} = \frac{|A|^2}{2 \rho_0 c_0 ^2} ; \qquad \overline{I_x} = c_0  \overline{e_{ac}} 
$$
où $I_x$ dénote l'intensité acoustique algébrique dans la direction $x$ (projection du vecteur intensité acoustique dans la direction $\vec{e}_x$).
}


\item 
\question{Comment faut-il modifier les relations données aux questions 2, 3 et 5 pour décrire une onde plane progressive monochromatique d'amplitude $B$ se déplaçant dans la direction $-\vec{e}_x$? }
\answer{
On peut partir d'une expression sous la forme 
$$
p'(x,t) = B e^{i(-k x - \omega t)} =   |B| \cos (-k x - \omega t+ \varphi_B) =  |B| \cos (-k( x + c_0 t)+ \varphi_B) 
$$
qui est bien de la forme $p'(x,t) = g(x+c_0 t)$.
Dans ce cas la perturbation de vitesse associée est 
$$u'(x,t) = -\frac{|B|\cos (-k( x + c_0 t)+ \varphi_B)}{\rho_0 c_ 0} =  -\frac{p'(x,t)}{\rho_0 c_ 0}.$$
La densité d'énergie moyenne et le flux d'énergie moyen sont alors :
$$
\overline{e_{ac}} = \frac{|B|^2}{2 \rho_0 c_0 ^2} ; \qquad \overline{I_x} = - c_0  \overline{e_{ac}} 
$$
}

\question{
\item {\em Question supplémentaire*}.
On reprend l'étude dans le cas d'un {\em solide élastique} de propriétés mécaniques $E$ (module d'Young), $\nu$ (coefficient de Poisson), $\rho_0$ (masse volumique).

Montrez qu'un milieu infini permet la propagation d'{\em ondes de compression} (qui généralisent les ondes acoustiques) avec une vitesse donnée par 
$$
c_l^2 = \frac{E(1-\nu)}{\rho (1+\nu)(1- 2 \nu)}.
$$
}

\section{Réflexion/transmission d'une onde acoustique}


\question{
On étudie la propagation d'ondes acoustiques dans un milieu composé de deux demi-espaces occupés par des fluides\footnote{Le cas ou l'un des milieux (ou les 2) est un solide élastique se traite de la même façon en remplaçant $c$ par $c_l$} de propriétés moyennes différentes, c'est-à-dire :
$$
(x<0 ) : \quad \rho_0 = \rho_1 ; c_0 = c_1 ; \quad \qquad (x>0 ) : \quad  \rho_0 = \rho_2 ; c_0 = c_2
$$
%On considère des ondes monodimensionelles de la forme : 
%$p' = p'(x,t)$ et $\vec{u'} = u'(x,t) \vec{e}_x$.

Dans les applications numériques, on considèrera que le milieu $1$ est de l'air ($\rho_1 = 1.225 kg/m^3 ; c_1 = 340 m/s$) et que le milieu $2$ est de l'eau 
($\rho_2 = 1000 kg/m^3 ; c_2 = 1500 m/s$).
}
\setcounter{enumi}{7}
 \item 
 \question{
On suppose que le champ de pression est donné par une loi de la forme suivante (ou il est sous-entendu qu'il convient de ne retenir que la partie réelle des quantités complexes) :
\begin{equation}
p'(x,t) = \left\{ \begin{array}{ll} 
A e^{i (k_1 x - \omega t)} + B e^{i (-k_1 x - \omega t)} & \quad ( \mbox{ pour }  x<0) \\
C e^{i (k_2 x - \omega t)} & \quad ( \mbox{ pour }  x>0) 
\end{array}
\right.
\label{eq:ABC}
\end{equation}

Comment peut-on qualifier les 3 ondes d'amplitudes $A$, $B$, $C$ apparaissant dans cette expression ? Quelle est l'interprétation physique de cette situation ?
}
\answer{
\begin{equation}
p'(x,t) = \left\{ \begin{array}{ll} 
A e^{i (k_1 x - \omega t)} + B e^{i (-k_1 x - \omega t)} & \quad ( \mbox{ pour }  x<0) \\
C e^{i (k_2 x - \omega t)} & \quad ( \mbox{ pour }  x>0) 
\end{array}
\right.
\label{eq:ABC}
\end{equation}
Du coté $x<0$ on a la superposition d'une onde simple d'amplitude $A$ se propageant à la vitesse $+k_1/\omega$ {\em (onde incidente)} et d'une onde simple d'amplitude $B$ se propageant à la vitesse $-k_1/\omega$ {\em (onde réfléchie)}.
Du coté $x>0$ on a reconnait une onde simple d'amplitude $B$ se propageant à la vitesse $+k_2/\omega$ {\em (onde transmise)} .
}

\item 
\question{
Quel est l'expression des paramètres $k_1$ et $k_2$ en fonction de la fréquence $\omega$ et des propriétés des milieux correspondants ?

Application numérique : donnez la valeur de $k_1$ et $k_2$ ainsi que les longueurs d'ondes associées $\lambda_1$, $\lambda_2$ pour une fréquence $\omega = 2 \pi f $ avec $f = 440 Hz$ (fréquence de la note de référence $la_4$ pour les instruments de musique) pour une transmission entre l'air et l'eau. 
}
\answer{
On peut noter tout d'abord que la pulsation $\omega$ est la même pour les 3 ondes considérées. Normal car on étudie un problème en régime harmonique en temps. La vitesse des ondes étant différente dans les 2 milieux, les nombre d'ondes sont donnés par 
$k_1 = \omega/c_1$ et $k_2=/\omega / c_2$.

Application numérique avec les données du problème :
$$
k_1 = 8.13 rad/m; \quad k_2 = 1.843 rad/m; \quad \lambda_1 = 0.773 m ; \quad \lambda_2 = 3.4 m$$
}






\item
\question{
A partir des équations du mouvement, donnez l'expression du champ de vitesse $u'(x,t)$ (respectivement pour $x<0$ et pour $x>0$) .
}
\answer{
En reprenant la démarche de la première partie et en l'adaptant à la situation considérée ici, on aboutit à :
\begin{equation}
u'(x,t) = \left\{ \begin{array}{ll} 
\frac{1}{\rho_1 c_1} \left( A e^{i (k_1 x - \omega t)} - B e^{i (-k_1 x - \omega t)} \right)& \quad ( \mbox{ pour }  x<0) \\
\frac{1}{\rho_2 c_2} \left( C e^{i (k_2 x - \omega t)} \right) & \quad ( \mbox{ pour }  x>0) 
\end{array}
\right.
\label{eq:ABCu}
\end{equation}
}

\item 
\question{
Justifiez que la pression $p'$ et la vitesse $u'$ sont continues à l'interface entre les deux milieux 
(en $x=0$). En déduire un système de 2 équations pour les 2 inconnues $B$ et $C$ ($A$ étant supposé connu).
}
\answer{Sous les hypothèses du milieu continu (satisfaites dans le cadre dans le cadre de l'acoustique linéaire), la pression et la vitesse sont des champs continus (et dérivables).
On en déduit :
\footnote{En toute rigueur, il faut tenir compte du fait que l'interface entre les deux milieux oscille autour de sa position moyenne. En notant $X'(t)$ le déplacement (lagrangien) de cette interface, celui-ci est relié à la vitesse par $d X'/dt = u'(X',t)$. Il faudrait donc écrire la condition de raccord pour la pression sous la forme $p'(X'_-,t) = p'(X'_+,t)$.  On peut faire un développement limité sous la forme $p'(0_-,t) + X' \left(\partial p'/\partial x\right)_{x=0_-} + ... =
 =p'(0_+,t) + X' \left(\partial p'/\partial x\right)_{x=0_+} + ... $. Cependant les déplacements étant supposés faibles, on peut négliger dans ce développement tous les termes proportionnels à $X'$ (et d'ordre plus élevé), ce qui finalement aboutit à écrire la continuité de la pression en $x=0$. De même pour la vitesse.} 
\begin{eqnarray}
p'(0-,t) = p'(0_+,t) 
& \Longrightarrow & A e^{-i  \omega t} + B e^{-i  \omega t} = C e^{-i  \omega t} 
\\
u'(0-,t) = u'(0_+,t) 
&\Longrightarrow & 
\frac{1}{\rho_1 c_1} \left( A e^{-i  \omega t} + B e^{-i  \omega t} \right) 
= \frac{C}{c_2 \rho_2} e^{-i  \omega t} 
\end{eqnarray}

En simplifiant par $e^{-i\omega t}$ on arrive donc à un système de deux équations à deux inconnues :

\begin{equation}
 \left\{ 
 \begin{array}{lcr}  A+B &=& C,\\
 \displaystyle \frac{A-B}{\rho_1 c_1} &=& \displaystyle  \frac{C}{\rho_2 c_2}.
 \end{array}
\right. 
\label{eq:ABCeqs}
\end{equation}

}




\item 
\question{En déduire la valeur des coefficients de réflexion et de de transmission en amplitude-pression :}
\answer{La résolution de ce système conduit directement au résultat attendu :}
\begin{eqnarray}
r = \frac{B}{A} &=& \frac{\rho_2 c_2 - \rho_1 c_1}{\rho_1 c_1 + \rho_2 c_2},\\
t = \frac{C}{A} &=& \frac{2 \rho_2 c_2}{\rho_1 c_1 + \rho_2 c_2},
\label{eq:rt}
\end{eqnarray}


\item \question{Montrez que les coefficients de réflexion et de transmission en énergie ont les expressions suivantes :}
\answer{D'après le cours et les résultats de la première partie, les flux d'énergie moyens $\overline{I}_x$ associés aux ondes incidente, réfléchie et transmise ont pour expression :
$$
\overline{I}_{x,A} = + \frac{|A|^2}{\rho_1 c_1} ; 
\quad \overline{I}_{x,B} = - \frac{|B|^2}{\rho_1 c_1} ;
\quad \overline{I}_{x,C} = + \frac{|C|^2}{\rho_2 c_2}.
$$
On en déduit :}

\begin{eqnarray}
R = \left| \frac{\overline{I}_{x,B}}{\overline{I}_{x,A}}  \right| &=& \left(\frac{\rho_1 c_1 - \rho_2 c_2}{\rho_1 c_1 + \rho_2 c_2}\right)^2,\\
T =\left| \frac{\overline{I}_{x,C}}{\overline{I}_{x,A}}  \right| &=& \frac{4 \rho_1 c_1\rho_2 c_2}{(\rho_1 c_1 + \rho_2 c_2)^2}.
\label{eq:RT}
\end{eqnarray}
\question{Vérifiez que $T+R = 1$. Quelle est l'interprétation physique de ce résultat ?}
\answer{ On a bien $T+R =1$, ce qui traduit la conservation de l'énergie : le flux  d'énergie incident est égal à la somme du flux transmis et du flux réfléchi.}

\item \question{Application numérique : donnez les valeurs de $t,r,T,R$ pour la transmission entre l'air et l'eau.
Exprimez $T$ en décibels.}
\answer{ Application numérique : on trouve 
$r = -0.9994$, $R = 0.9989$, $t=1.9994$, $T = 0.0011$.
On peut remarquer que $r$ est proche de $-1$ qui correspond au coefficient de réflexion sur un mur rigide (cf cours). $t$ est proche de $2$, il y a donc transmission d'une onde d'amplitude a peu près double de l'amplitude incidente, par contre le flux d'énergie associé (mesuré par $T$) est très faible.
Mesuré en décibels, on a : 
$$
T_{db} = 10 \log_{10} (T) \approx -29.54 db.
$$
}

\question{
\item {\em Question supplémentaire*}.
On considère maintenant, non plus un milieu infini des deux côtés de la discontinuité, mais un conduit cylindrique d'axe $\vec{e}_x$, dont la section varie brutalement entre une valeur $S_1$ (pour $x<0$) et une valeur $S_2$ (pour $x>0$). On suppose que le champ de pression reste donné par la description (\ref{eq:ABC}).

On note $q'(x,t)  = u' S$ le débit de volume à travers la section du tube.

\begin{enumerate}

\item A partir des expressions de la question 10, donnez  l'expression de $q'(x,t)$, respectivement  pour $x<0$ et $x>0$.

\item
Montrez qu'a partir de la continuité de la pression et du débit en $x=0$ on obtient des formules qui généralisent les équations (2,3,4,5) %(\ref{eq:rt} et \ref{eq:RT}) 
en remplaçant $\rho_1 c_1$ par $Z_1 = \rho_1 c_1/S_1$ et  $\rho_2 c_2$ par $Z_2 = \rho_2 c_2/S_2$. ({\em $Z_1$ et $Z_2$ sont les impédances spécifiques des deux portions de tube}).
\item

Donnez la valeur de $T$ (en décibels) pour $S_1 = 0.6 cm^2$ et $S_2= 0.03 cm^2$.

\end{enumerate}
}

\section{Application au fonctionnement de l'oreille moyenne}

\begin{figure}
\includegraphics[width=.45\linewidth]{OreilleMoyenne.jpg}
\includegraphics[width=.45\linewidth]{ModeleOreille.png}
\caption{$(a)$ physiologie de l'oreille et $(b)$ Modèle mécanique simplifié des osselets de l'oreille moyenne.}
\end{figure}
\question{
Le rôle de l'oreille moyenne est d'optimiser la transmission des ondes acoustiques entre un milieu aérien (le conduit auditif) et un milieu aqueux (la cochlée ou oreille interne).
Pour cela, la solution sélectionnée par la sélection naturelle a consisté à utiliser 3 osselets constituant une sorte de levier reliant ces deux milieux. 
La figure ci-dessus représente ce modèle mécanique de manière très schématique.

On donne les dimensions des bras de levier : $d_1 = 1,3cm$, $d_2 = 1cm$, 
ainsi que la surface du tympan $S_1 = 0.6cm^2$ et celle de la fenêtre ovale $S_2= 0.03cm^2$.
}


Appelons $\cal S$ le système mécanique constitué par les osselets, comme schématisé sur la figure. Notons $O$ le point pivot (supposé fixe), $M_1$ le point situé au centre du tympan,
$M_2$ le point situé au centre de la fenêtre ovale. On introduit une base ($\vec{e}_x,\vec{e}_y;\vec{e}_z$ tel que la direction $x$ est perpendiculaire aux surfaces du tympan et de la fenêtre ovale (direction verticale sur la figure), et la direction $y$ est alignée avec les leviers (direction horizontale sur la figure). 
 
 \setcounter{enumi}{15}
\item 
\question{Donnez la relation cinématique entre $u_1$ et $u_2$ (vitesses au niveau du tympan et de la fenêtre ovale).}

\answer{
Le solide a un mouvement de corps rigide avec un vecteur rotation $\vec {\Omega}$. Les vitesses des points $M_1$ et $M_2$ s'en déduisent par (avec les notations classiques de mécanique des solides) : 

$$
\vec{V}_{M_1 \in {\cal S}} = \vec{\Omega} \wedge \vec{OM_1} ; \quad \vec{V}_{M_2 \in {\cal S}} = \vec{\Omega} \wedge \vec{OM_2}.
$$

En notant $u_1$ et $u_2$ les valeurs algébriques des vitesses des points $M_1$ et $M_2$ (comptées dans la direction $x$), $\Omega_z$ le taux de rotation, et $d_1,d_2$ les bras de leviers, ces relations conduisent à $u_1 = \Omega_z d_1$ et $u_2 = \Omega_z d_2$, d'où l'on déduit en supprimant l'inconnue $\Omega_z$ :
\begin{equation}
\frac{u_1}{d_1} = \frac{u_2}{d_2}.
\end{equation}
}

\item 
\question{ \underline{ En négligeant l'inertie des osselets}, donnez une relation entre les forces $F_1$ et $F_2$  puis entre les pressions $p'_1$ et $p'_2$.}

\answer{
En notant $\bf{M}$ la matrice d'inertie du solide $\cal S$, le théorème du moment dynamique s'écrit : ${\bf M} \cdot d \vec{\Omega}/dt = \vec{OM_1} \wedge \vec{F_1} + \vec{OM_2} \wedge \vec{F_2}$ avec $\vec{F_1} = p_1 S_1 \vec{e_x}$ et $\vec{F_2} = -p_2 S_2 \vec{e_x}$ les forces exercées sur chacune des surfaces.
Si l'on néglige l'inertie, l'équilibre des moments conduit donc à :

\begin{equation}
d_1 S_1 p_1  = d_2 S_2 p_2.
\end{equation}
}



\item 
\question{Reprendre l'étude de la partie précédente, et donnez une expression du coefficient 
de transmission.

Comparez avec les résultats des questions 10 et 11.


Pour en savoir plus :
\verb| http://www.cochlea.eu/oreille-generalites/oreille-moyenne |

}

\answer{
On reprend donc l'étude de la partie précédente en remplaçant $u_1$ par $u'(0_-,t)$, $u_2$ par $u'(0_+,t)$, $p_1$ par $p'(0_-,t)$ et $p_2$ par $p'(0_+,t)$. On considère deux nouveau une onde incidente $A$, une onde réfléchie $B$ et une onde transmise $C$. 
On aboutit à un sytème de deux inconnues :  

\begin{equation}
 \left\{ 
 \begin{array}{lcr}  \displaystyle \frac{1}{\rho_1 c_1 d_1} (A-B) &=&\displaystyle  \frac{1}{\rho_2 c_2 d_2} C,\\
 \displaystyle d_1 S_1 (A+B)&=& \displaystyle  d_2 S_2 C
 \end{array}
\right. 
\label{eq:ABCeqs}
\end{equation}
La résolution conduit à :
$$
t = \frac{C}{A} = \frac{2 S_1 \rho_2 c_2 d_1 d_2}{ \rho_1 c_1 S_1 d_1^2 + \rho_2 c_2 S_2 d_2^2 } 
$$

Pour en déduire le coefficient de transmission en énergie, il faut prendre garde au fait que le flux d'énergie $\cal P$ (puissance acoustique, en $W$)  est égal à l'intensité acoustique 
$\overline{I}_x$ (puissance acoustique surfacique en $W/m^2$) multiplié par la section des conduits correspondants. Ce qui conduit, pour l'onde incidente et l'onde transmise, aux formules suivantes :

$$
{\cal P}_{A} = S_1 \overline{I}_{x,A} = \frac{S_1}{\rho_1 c_1} |A|^2; \quad
{\cal P}_{C} = S_2 \overline{I}_{x,C} = \frac{S_2}{\rho_2 c_2} |C|^2.
$$

On en déduit finalement :

$$
T = \frac{{\cal P}_{C}}{{\cal P}_{A}} =
\frac{ 4\rho_1 c_1 S_1 d_1^2 \rho_2 c_2 S_2 d_2^2 }{(\rho_1 c_1 S_1 d_1^2 + \rho_2 c_2 S_2 d_2^2 )^2} 
$$

L'application numérique donne alors $T = 0.0368$, soit $T_{dB} = -14.3 dB$.
Par rapport à la transmission directe (partie 2) la chaîne d'osselets permet donc de gagner environ $15 dB$ !
}
  
\end{enumerate}



 




%%%%%%%%%%%%%%%%%%%%%%%%%%%%%%%%%%%%%%%%%%%%%%%%%%%%%%%%%%%%%%%%%%%%%%%%%%%%%%%%%%%%%%%%%%%%%%%%%%%
\end{document}
%%%%%%%%%%%%%%%%%%%%%%%%%%%%%%%%%%%%%%%%%%%%%%%%%%%%%%%%%%%%%%%%%%%%%%%%%%%%%%%%%%%%%%%%%%%%%%%%%%%

