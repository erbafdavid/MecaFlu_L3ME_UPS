
% Style des vecteurs : fleche ou gras ?

%\newcommand{\myvec}[1]{\boldsymbol{#1}}
\newcommand{\myvec}[1]{\vec{#1}}

\newcommand{\mytensor}[1]{\accentset{\Rightarrow}{#1}} % needs \usepackage{accents}

%---------------------------
% Operateurs differentiels :
%---------------------------

\newcommand{\divergence}{\mbox{\rm div}\,}

\newcommand{\gradient}{\myvec{\mbox{\rm gra}}\mbox{\rm d}}
% \newcommand{\gradient}{\mathbf{grad}\,}
% \newcommand{\ggradient}{\stackrel{\Rightarrow}{\mbox{gra}}\!\!\!\,\mbox{d}\,}
\newcommand{\ggradient}{\accentset{\Rightarrow}{\mbox{\rm gra}}\mbox{\rm d}\!}

%\renewcommand{\dot}[1]{\accentset{\hbox{\huge .}}{#1}}
\newcommand{\mydot}[1]{\accentset{\centerdot}{#1}}

\newcommand{\rot}{\vec{\mbox{\rm ro}}\mbox{\rm t}\,}
%\newcommand{\rot}{\mathbf{rot}\,}

% \newcommand{\vnabla}{\vec{\nabla}}
\newcommand{\vnabla}{\boldsymbol{\nabla}}

% Fonctions speciales:

\newcommand{\besselj}[1]{\mbox{J}_{#1}}
\newcommand{\besselk}[1]{\mbox{K}_{#1}}
\newcommand{\bessely}[1]{\mbox{Y}_{#1}}
\newcommand{\besseli}[1]{\mbox{I}_{#1}}

% Vecteurs, tenseurs et torseurs:

\newcommand{\ex}{\myvec{e}_{x}}
\newcommand{\ey}{\myvec{e}_{y}}
\newcommand{\ez}{\myvec{e}_{z}}

\newcommand{\er}{\myvec{e}_{r}}
\newcommand{\erho}{\myvec{e}_{\rho}}
\newcommand{\ephi}{\myvec{e}_{\varphi}}
\newcommand{\etheta}{\myvec{e}_{\theta}}

%\newcommand{\tensor}[1]{\stackrel{\Rightarrow}{#1}}
\newcommand{\tensor}[1]{\mbox{\sl \textbf{#1}}}
\newcommand{\torseur}[4]{
   \!\!\!\! \left . \begin{array}{c} \\ \\ _#1 \end{array} \!\!\!
   \right \{ \!\!\!
   \begin{array}{#4} #2 \\ \\ #3 \end{array}}

% Integrales multiples:

\newcommand{\odblint}[1]{\int\!\!\!\!\!\int_{#1} \hskip -7mm \bigcirc \;}
\newcommand{\dblint}{\int\!\!\!\!\!\int}
\newcommand{\tplint}{\int\!\!\!\!\!\int\!\!\!\!\!\int}

% Fractions:

\renewcommand{\dfrac}[2]{\displaystyle \frac{#1}{#2}}

% Derivees ordinaires et partielles:

\newcommand{\dpdt}[1]{\dfrac{\partial #1}{\partial t}}
\newcommand{\dpdx}[1]{\dfrac{\partial #1}{\partial x}}
\newcommand{\dpdy}[1]{\dfrac{\partial #1}{\partial y}}
\newcommand{\dpdz}[1]{\dfrac{\partial #1}{\partial z}}

\newcommand{\ddpdt}[1]{\dfrac{\partial^2 #1}{\partial t^2}}
\newcommand{\ddpdx}[1]{\dfrac{\partial^2 #1}{\partial x^2}}
\newcommand{\ddpdy}[1]{\dfrac{\partial^2 #1}{\partial y^2}}
\newcommand{\ddpdz}[1]{\dfrac{\partial^2 #1}{\partial z^2}}

\newcommand{\dpdr}[1]{\dfrac{\partial #1}{\partial r}}
\newcommand{\dpdrho}[1]{\dfrac{\partial #1}{\partial \rho}}
\newcommand{\dpdphi}[1]{\dfrac{\partial #1}{\partial \varphi}}
\newcommand{\dpdtheta}[1]{\dfrac{\partial #1}{\partial \theta}}

\newcommand{\ddpdr}[1]{\dfrac{\partial^2 #1}{\partial r^2}}
\newcommand{\ddpdrho}[1]{\dfrac{\partial^2 #1}{\partial \rho^2}}
\newcommand{\ddpdphi}[1]{\dfrac{\partial^2 #1}{\partial \varphi^2}}
\newcommand{\ddpdtheta}[1]{\dfrac{\partial ^2#1}{\partial \theta^2}}

\newcommand{\ddt}[1]{\dfrac{d #1}{dt}}
\newcommand{\ddx}[1]{\dfrac{d #1}{dx}}
\newcommand{\ddy}[1]{\dfrac{d #1}{dy}}
\newcommand{\ddz}[1]{\dfrac{d #1}{dz}}
\newcommand{\ddr}[1]{\dfrac{d #1}{dr}}

\newcommand{\ddtref}[2]{\dfrac{d #1}{dt}_{\! | #2 }}
\newcommand{\dpdtref}[3]{\dfrac{\partial #1}{\partial #2}_{\! | #3 }}

% Misc:

\newcommand{\bareme}[1]{\reversemarginpar \marginpar{%
	\hspace{8mm} \small \makebox[5mm]{/#1}% 
	\begin{picture}(0, 0)\put(0, 0.5){\makebox[12mm]{\dotfill}}\end{picture}%
	}}

\newcommand{\mycaption}[1]{\caption{\sl #1}}

\newcommand{\ligne}[1]{\hrule height #1\linethickness \hfill}

\newcommand{\thickline}[2]{\linethickness{#1} \line(1, 0){#2}}

\newcommand{\myline}{\noindent\underline{\hspace{\textwidth}}}
\newcommand{\mysection}[1]{\vskip 0.5cm \section{#1}\vskip -1.4cm 
   \myline \vskip 0.4cm \myline \bigskip}

\newcommand{\etal}{\textit{et al.}}

\newcommand{\varray}[1]{\renewcommand{\arraystretch}{#1}}

\newcommand{\puissance}[1]{^{\mbox{\footnotesize #1}}}
\newcommand{\indice}[1]{_{\mbox{\footnotesize #1}}}

%---------------------------------------------------------------------
% New environments:
%---------------------------------------------------------------------

\newcounter{MyEnumCounter}
\newcounter{MySaveCounter}
\newenvironment{MyEnum}{%
  \begin{list}{\arabic{MyEnumCounter}.}{\usecounter{MyEnumCounter}%
  \setcounter{MyEnumCounter}{\value{MySaveCounter}}}
  }{%
  \setcounter{MySaveCounter}{\value{MyEnumCounter}}\end{list}%
}
\newcommand{\MyEnumReset}{\setcounter{MySaveCounter}{0}}

\newenvironment{deuxcols}{\begin{tabular}{lr} \hspace*{-9.7mm}}{\end{tabular}}

\newenvironment{dem}{\noindent %
   \begin{tabular}{||l} \textsl{D\'emonstration :} \\ % 
   \begin{minipage}{15.5cm} \footnotesize} %
   {\end{minipage}\end{tabular}}

\newenvironment{abst}{\begin{quotation}\sl}{\end{quotation}}

\newenvironment{eqnbox}{\begin{equation}\begin{array}{|c|}  \hline \\ 
   \displaystyle}{\\ \\ \hline \end{array} \end{equation}}

% JFM symbols:

\DeclareMathSymbol{\varGamma}{\mathord}{letters}{"00}
\DeclareMathSymbol{\varDelta}{\mathord}{letters}{"01}
\DeclareMathSymbol{\varTheta}{\mathord}{letters}{"02}
\DeclareMathSymbol{\varLambda}{\mathord}{letters}{"03}
\DeclareMathSymbol{\varXi}{\mathord}{letters}{"04}
\DeclareMathSymbol{\varPi}{\mathord}{letters}{"05}
\DeclareMathSymbol{\varSigma}{\mathord}{letters}{"06}
\DeclareMathSymbol{\varUpsilon}{\mathord}{letters}{"07}
\DeclareMathSymbol{\varPhi}{\mathord}{letters}{"08}
\DeclareMathSymbol{\varPsi}{\mathord}{letters}{"09}
\DeclareMathSymbol{\varOmega}{\mathord}{letters}{"0A}

% ---------------------------------------------------------------------
% MISC SYMBOLS :
% ---------------------------------------------------------------------

\font\SY=msam10 
\def\carreblanc{\hbox{\SY \char'3}}
\def\carrenoir{\hbox{\SY \char'4}}
\def\diamblanc{\hbox{\SY \char'6}}
\def\diamnoir{\hbox{\SY \char'7}}
\def\triblancright{\hbox{\SY \char'102}}
\def\triblancleft{\hbox{\SY \char'103}}
\def\triblancup{\hbox{\SY \char'115}}
\def\triblancdown{\hbox{\SY \char'117}}
\def\trinoirright{\hbox{\SY \char'111}}
\def\trinoirleft{\hbox{\SY \char'112}}
\def\trinoirup{\hbox{\SY \char'116}}
\def\trinoirdown{\hbox{\SY \char'110}}
\def\rondblanc{\hbox{\scriptsize $\bigcirc$}}
\def\rondnoir{\hbox{\LARGE $\bullet$}}

\font\BB=msbm10 scaled 1095
\def\setr{\hbox{\BB R}}
\def\setc{\hbox{\BB C}}
\def\setn{\hbox{\BB N}}
\def\setz{\hbox{\BB Z}}

% Pour enlever la numerotation des pages de la table des matieres:

%%%% debut macro, a placer dans preambule %%%%
\makeatletter
\def\addcontentsline@toc#1#2#3{%
   \addtocontents{#1}{\protect\thispagestyle{empty}}%
   \addtocontents{#1}{\protect\contentsline{#2}{#3}{\thepage}}}
\def\addcontentsline#1#2#3{%
  \expandafter\@ifundefined{addcontentsline@#1}%
  {\addtocontents{#1}{\protect\contentsline{#2}{#3}{\thepage}}}
  {\csname addcontentsline@#1\endcsname{#1}{#2}{#3}}}
\makeatother
%%%% fin macro %%%%

\newcommand{\titre}[1]{ %
  \medskip \noindent \underline{\makebox[\textwidth][l]{\textbf{#1}\textcolor{white}{pl}}}}% \\}

\newcommand{\sstitre}[1]{ %
  \bigskip \centerline{\textbf{#1}} \smallskip}

\def\draft{\overfullrule 5pt} % The \draft command marks the overful boxes

\def\indentlist{\list%
        {}{\labelwidth 0pt \leftmargin 3\labelsep}}
\let\endindentlist\endlist \relax

\def\datelist{\list%
        {}{\settowidth\labelwidth{[2001/02 :]}
        \leftmargin\labelwidth
        \advance\leftmargin\labelsep}
}
\let\enddatelist\endlist \relax

\def\longuelist{\list%
        {}{\settowidth\labelwidth{[Etablissement :]}
        \leftmargin\labelwidth
        \advance\leftmargin\labelsep}
}
\let\endlonguelist\endlist \relax

\def\shortlist{\list%
        {}{\settowidth\labelwidth{$\bullet$}
        \leftmargin\labelwidth
        \advance\leftmargin\labelsep}
}
\let\endshortlist\endlist \relax

